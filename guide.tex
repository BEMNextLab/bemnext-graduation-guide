\documentclass{latex-format/stylesheets/BEMNextstyle}
\begin{document}

\pagestyle{trail}
\mainmatter
\cleardoublepage
\phantomsection
\addcontentsline{toc}{chapter}{BEMNext Lab Guide to Graduation}
\chapter*{BEMNext Lab Guide to Graduation}
Draft, Version 0.0.2\\
BEMNext Lab, TU Delft, and contributors

\section*{Introduction}
This document gives guidenance instructions for graduation at the BEMNext Lab.  Please note that if this guide is in any way conflicting with the formal regulations of the university or faculty, the latter take precedence over this guide.

\subsection*{This document}
This document can be downloaded from the BEMNext Lab website (\url{http://www.bemnext.org}). The source can be downloaded from Github (\url{https://github.com/jeroencoenders/bemnext-graduation-guide}). The source is released under the MIT license (\url{http://opensource.org/licenses/MIT}). If you have comments or find mistakes, please correct them and submit them to the Github repository or file an issue. We hope that in this way this document will develop into the perfect graduation guide.

\section*{Starting the Master's thesis process}
You are almost done with your regular courses and ready to start your final project? Great! If you have some courses left to finish, we highly recommend to finish them first as coursework and the thesis work often interferes. It of course does not hurt to inform yourself about research themes and available topics.

\subsection*{Looking for a topic?}
So, you are ready to start looking for a topic? The BEMNext Lab has many graduation themes and topics. Take a look at our website (\url{http://www.bemnext.org}). It contains general information on our lab and the areas of research we are active in. Generally speaking, many of our graduation topics will be in these areas as well.\\
The first thing you will need to do next is the research programme leader of the BEMNext Lab, dr.ir. Jeroen Coenders, by sending him an e-mail (j.l.coenders@tudelft.nl). He will make an appointment with you to discuss topics with you which match your interests, knowledge and ambitions. We strongly believe that if you choose a topics close to your interests and passions, the results will be better, because you will be more motivated. You will need to spend researching in this field for more than half a year after all.\\
Of course it is also possible to bring your own topic or a topic that a company is providing you with (please refer to the section on 'Involving companies' for more information).\\
Usually we leave it to you to define the exact assignment based on the topic so that you are put your own stamp on it, but if you prefer the supervisor(s) can define the assignment for you.

\subsection*{Managing the project}
We believe in a model where you are largely in control over the management of your thesis work. This means you will get a lot of freedom to manage yourself and the committee. Please note that the supervisor will keep an eye on you, but that the progress of your work is largely your own responsibility. Half a year sounds like a long time in the beginning, but after half a year you will probably find yourself questioning where that half year went. Try to take control over the planning and management of your time and manage your meetings well. If you need any help talk to your daily supervisor or other committee members.

\subsection*{Choosing the daily supervisor}
In most cases where you choose a topic of the BEMNext Lab, we require that you will pick your daily supervisor at the BEMNext Lab. He or she will be able to help you with most of the issues you might encounter. Please note that a daily supervisor does not mean that you will work on a daily basis with him or her (but it can happen), he or she is just your first point of contact.\\
Please note that we do not appreciate that you migrate a topic away from our lab. We have put a lot of hard in the research and development of these fields and would like to keep them part of our lab.

\subsection*{Choosing the committee}
Together with your daily supervisor, you will select your professor and a graduation commitee. There are specific rules for the committees. Please make sure that you look into them. We recommend that you mainly pick domain experts who are able to help you with a specific piece of your work or bring specific knowledge to the table.\\
Sometimes topics will be linked to a (partially) fixed committee as these people are part of a research team that looks at this topic on a larger scale, such as a research project. These people will also be the best people to support you. 

\subsection*{Involving companies}
Some topics are linked to companies who choose to participate in a research programme or project and these companies will provide committee members for your project. As these companies are involved in the research and often providing detailed knowledge and support, they are very suitable to support your project. Sometimes it might even be possible to do an internship or obtain a graduation project position with this company.\\
In case you are interested in involving a company, because you want to do an internship with them, or because they provide you with a topic, discuss this with the research programme leader first. We strictly enforce the rule that a company is only allowed to join the project and committee in case they provide a significant input in the project (in the form of knowledge, experience, data, equipment, etc.). We do not allow companies to obtain 'free rides' on research results through graduate students. Unfortunately we have had bad experiences in the past. The university needs to funds its research by government or company support and we cannot allow our knowledge to flow away through graduation projects at minimal cost for the company. The only exception we make is if a company is willing to make an (significant) invesment in the BEMNext Lab or our research projects by funding us. In case of doubt, please discuss this with the research programme leader.

\subsection*{The project plan}
This section describes the project plan. This is the first report you will be submitting to your committee and it is required for your start-up meeting. Make sure that you think this plan through, but also do not spend forever on it as it will change over the course of time. Study enough material so that you have a good overview over the problem.

Typical content:
\begin{enumerate}
\item Introduction in the field
\item Problem analysis
\item Plan of work
\item Outline
\item Planning
\end{enumerate}

Please include in the appendices 
\begin{itemize}
\item your contact details
\item student number
\item the contact details of your committee
\item other relevant information. 
\end{itemize}

\subsubsection*{Problem analysis}
In the problem analysis you study and analyse your problem by asking research questions and setting objectives. Please make them specific and answerable in your thesis. In the end ideally your conclusions must give an answer to your research questions.

\subsubsection*{Plan of work}
In this section you describe how you will be answering your research questions and how you will achieve your objectives. You describe this by briefly explaining each step of the process. Of course, you cannot look into the future and can imagine everything on beforehand. You explain what you know and how you imagine things will go. The plan of work is mainly there to help you. We won't hold it against you when you change the plan along the way in liaison with the committee. Include a planning in the appendices of the report in line with your plan of work. For more information also refer to the section on Planning.

\subsubsection*{Final report outline}
Include your envisioned outline in your plan. Most scientific reports follow more or less the same structure. In case you are doing any experiments, you might modify the outline, but note usually the outline more or less looks like this:

\begin{enumerate}
\item Introduction
\item Methodology
\item Experiment design
\item Experiment results
\item Validation
\item Discussion
\item Conclusions
\item Recommendations
\end{enumerate}

\section*{Planning}
\label{sec:planning}
Note that a planning is something for you rather than the committee. It is an instrument for you to measure your progress against your original plan. A pure sequential planning is suspicious, because you are always doing multiple things in parallel. So, please plan this as realistic as possible. Don't forget to include meetings in your planning. Do a backwards planning exercise from the date you are envisioning to be graduating towards the start. This will often give valuable insights in the amount of time you realistically need.

\subsection*{Start-up meeting}
During the start-up meeting you will meet with your committee as a whole for the first time. We will try to get a clear idea about what you will do to graduate. You present your plan of work and the committee will give additional information. Try to end the meeting with a clear conclusion. This will help you to keep on track during the process of the graduation project and helps to manage expectations.

\subsection*{Progress meetings}
During the progress meetings the committee will measure your progress and help by guiding you in the right direction. It is recommended to make clear presentations on your progress for each meeting as this helps guide the discussions in the right direction.

\subsection*{The 'Go/No-Go' or 'Green Light' meeting}
The final meeting that you will attend is the 'go/no-go' or 'green light' meeting. This meeting will take place when you submit your concept report. This report is the final version as if it was finished for you. So it needs to contain all content, images, lay-out, etc. As a result of this meeting the committee will conclude if you are ready for your final presentation or that you need additional work to reach this goal.

\subsection*{The final committee meeting}
The final committee meeting takes place without you about hour before you will have your final presentation. In this meeting the committee will discuss and assess your work. This will result in a preliminary conclusion that you can influence with your performance at your final presentation. 

\subsection*{The final presentation}
The final presentation usually takes place 2 to 3 weeks after the 'Go/No-Go' meeting if the committee has approved your concept report.

Your presentation will take about 20 minutes after which you will answer questions from the committee and audience. Please note that these all are part of the final assessment of your work.

If there are no irregularities, the chairman of your presentation will announce the verdict of the committee on your performance after which (in case of success you will have completed the thesis work. In case you have completed your entire programme, you can apply for your degree.

\section*{Writing the Master's thesis}
\subsection*{Writing}
Your Master's thesis will need to be written in a professional and scientific manner.

\subsection*{Designing software}
In case you are designing a piece of software, we would like you to know a few things.

\subsubsection*{Design diagrams}
We would like to make sure that other scientifist are able to understand your software design. Please use existing diagram methods to visualise your design.

Some good techniques are:
\begin{itemize}
\item Object structure: UML;
\item Processes: Workflow diagrams, Program Structure Diagrams (PSD, only for small applications)
\end{itemize}

\subsubsection*{Pseudo-code}
We strongly discourage you to copy-and-paste source code in your report. In case you would like to explain an algorithm, conside the use of pseudo-code.

\subsubsection*{User manual}
Please include a user manual in your appendices.

\subsubsection*{The source code}
Do not include the code in your appendices. It is a waste of paper. Please make sure you hand in the code at your supervisor.

\subsection*{Referencing}
At the BEMNext Lab we prefer the Harvard style of referencing.

\subsection*{\LaTeX}
At the BEMNext Lab we strongly recommend the use of~\LaTeX to write your thesis. The major advantages of ~\LaTeX are that~\LaTeX can produce 
\begin{itemize}
\item Proper formatting of large documents;
\item Easy, hassle-free referencing;
\item Automated citation;
\item Easy inclusion of formulas and equations;
\item Easy support for lists and numbered lists;
\end{itemize}

Main advantage over most word processors (such as Microsoft Word)  is the stability when your document gets larger. Often word processors corrupt files when the document becomes larger, leading to a lot of unnecessary rework, or corrupt reference links, etc. Main advantage over most desktop publishing applications is that~\LaTeX has been produced for scientific writing, so has excellent support for formulas and equations, images, figures, citation, etc. which usually in the case of a desktop publishing application is manual work.
The drawbacks of~\LaTeX compared to these applications is that it has a bit of a learning curve to lean the language and tools and that you are more restricted to manually tweak your lay-out.
However, at the BEMNext Lab we have made your live easy by providing you with a~\LaTeX format that can be used directly to produce your thesis.

Some people dislike the lay-out of~\LaTeX, because in their opinion it looks old-fashioned. We hope to produce with this document written in~\LaTeX that we have proven that that is not the case.

Please note that if you choose another way of producing your report, we do not take the inefficient time of fixing your files or manual formatting your report into account. If you spend extra time because of this, we assume this time is your own free time, but we strongly recommend to use your free time in a more efficient manner.

\subsubsection*{Format}
We have set a preformatted thesis to work on in the directory 'latex-format'.

\subsubsection*{Tools}
This section covers some different tools for different operating systems.

\paragraph*{\LaTeX on Windows}
If you are a Windows user, we recommend MikTeX in combination with an editor like TeXniccenter that can also compile your documents for your.

\paragraph*{\LaTeX on the Mac}
On the Mac MacTex provides a full toolkit for writing and compiling~\LaTeX. TeXworks is a useful editor for editing your files.

\paragraph*{\LaTeX on the Linux}
Linux often comes with free tools to write and compile~\LaTeX reports. Please refer to the documentation of your distribution to find out what the right tool is for your distribution.

\subsubsection*{Troubleshooting}
In this section we cover some trouble you might experience when using~\LaTeX.

\paragraph*{Missing html.sty}
When using the default installation of MacTeX you might experience an error message that "html.sty is missing". To resolve this, download the file at this location:~\url{http://www.tex.ac.uk/ctan/support/ltx2x/html.sty} and save it in the following directory: ~/Library/texmf/tex/latex/misc. You might have to create this directory.

\subsection*{Common mistakes}
In this section we list some common mistakes we have seen in the past. We list them here so that you can prevent them:
\begin{itemize}
\item References to figures, tables, equations, etc. are made starting with a capital, e.g. ``In Figure 1 can be seen that... '' or ``Equation 2 proves that ...''.
\item When an idea is not yours, you should reference the orginal author. In case you don't, you are running the risk of being accussed of plagiarism. Use citations to prevent this from happening.
\item Try to prevent claims that you cannot make. Words like ``all ...'', ``the most important ...'', ``the reason for ...'' are suspicious. You might want to replace them with ``one of the most important''
\end{itemize}

\subsection*{Good luck!}
Good luck with your project! If you have any questions and/or remarks, contact the BEMNext Lab (\url{http://www.bemnext.org})/

\end{document}